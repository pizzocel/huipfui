\section{huipfui}
\label{huipfui}

This is the HUman Interface Pipe Framework Under Implementation.

\subsection{Overview}

This project was started when I arrived in a holidays flat. I had a
Raspi in my bag, which I intended to attach to the TV set in order to run
the multimedia software \href{http://kodi.tv/}{Kodi}. 
But then I noticed I had forgotten to take a keyboard or mouse with me. 
So what could I do? At least, I
had the keyboard of the notebook. Could this keyboard be used to
control the Raspi?

%% In the next section add links to the software tools mentioned
There are readily available solutions 
\href{http://github.com/dottedmag/x2x}{x2x} and 
\href{http://synergy-project.org/}{synergy}, that allow to
control another computer over the network. But these require an X server
on Linux and on Raspis Kodi runs directly on the frame buffer. So these
solutions were of no use for me. I also refuted remote desktop tools as
pushing video through the network seemed to be too much overhead. If your
keyboard and mouse are USB devices, you also might consider 
\href{http://usbip.sourceforge.net/}{usbip}, which
can connect USB devices over IP networks. But my notebook keyboard was
not attached via USB.

However, linux seems already well prepared for replicating 
user input devices. 
A a matter of fact, you can create virtual
input devices with the uinput driver, which is usually found as
/dev/uinput or /dev/input/uinput depending on distribution. So the
simple idea is to read input events from the real hardware on the
notebook and feed it to the uinput on the Raspi.

Thus, the solution appears to be simple.  We only need a program that
can read the input events from the real hardware devices and another
one to write them to /dev/uinput.  The program to read the events is
called hid2out.  It sends representations of the events to standard
output.  These can then be fed to other program named in2hid

%%Add markup for the program names
In the configuration file you can select the input devices by
listing their paths.
You also can configure for each event, the action it should trigger.
So apart from the default of writing its representation to the 
standard output, you can determine, that a modified event should be written,
that the event should be ignored, that a process should be startd
or that hid2out should be terminated.
events

As in2hid creates an input device in Linux which cannot be distinguished
from real hardware input devices, this project can be used independent
of any desktop or GUI environment. In contrast to other solutions
you don't need X11. You can use it with KDE or Gnome and then switch to 
text console and still continue using it. Of course, it works with 
software that used the frame buffer directly like Kodi on Raspberry Pi.

\subsection{Building}
You've everything you need,
if you have the usual Linux software development tools installed
and the library libconfig with its development files.
On Debian it should suffice to enter the following line to
get all necessary packages:
\begin{verbatim}
apt-get install build-essential libconfig-dev
\end{verbatim}

In the directory cloned from the repository of
this project simply run the following commnand:
\begin{verbatim}
make
\end{verbatim}

\subsection{Installing}
Place the two executables 
\texttt{hid2out} and \texttt{in2hid}
built in the previous section
into a directory which makes calling
them most convenient for you. 
On most distribution \verb|/usr/local/bin| is a nice place.

Make sure that you can access the input devices that you want to 
be read by hid2out. Input devices are usually accessible by the 
group called \verb|input|. So the user starting hid2out 
has to be (made) a member of that group. Suppose her user name 
is Alice. Then you may execute  in your shell:
\begin{verbatim}
adduser alice input
\end{verbatim}
Before actually running hid2out a configuration file is needed.
How to write one is explained in the next section.

The user running in2hid needs the uinput device node with
write  access to it.
The node will be created automatically 
in \verb|/dev/uinput|
resp.~\verb|/dev/input/uinput|, when the module 
uinput is loaded. 
So you may want to add the module name \texttt{uinput} to the 
list of modules to be loaded at boot time.
On current Debian system you would
add a file to the directory \verb|/etc/modules-load.d|
containing just one line:
\begin{verbatim}
uinput
\end{verbatim}
In most other cases you can add the line to the file
\verb|/etc/modules|.

Most often you may already find a device node \verb|/dev/uinput|
even before the module is loaded. 
Then your system will probably load the module automatically
by on demand loading, when accessing \verb|/dev/uinput|
for the first time.
But we discourage to make use of on demand loading as
this would require to configure the permissions twice,
once for the time when the node is created and a second
time for when the module is loaded. If these settings
differ, you will note, that \verb|/dev/uinput| changes permissions
at the time it is accessed the first time.

%Usually this device is owned by root and only accessible by
%root. 
%Giving access to this device is rather dangerous, as
%an attacker may generate arbitrary input on your system and
%thus do nearly anything a user sitting in front of the machine
%can do. 
For security reasons you may choose to have only
one particular user allowed to access uinput.
Let's call him in2hid and create his account:
\begin{verbatim}
adduser uinput --disabled-password --shell /usr/local/bin/in2hid
\end{verbatim}

Now we need to set the permissions of \verb|/dev/uinput|.
The udev service takes care of all nodes, when they are created
by loading their modules.
A rule that makes it set permission to out needs could be:

\begin{verbatim}
KERNEL == "uinput", OWNER = "uinput", GROUP = "uinput", MODE = "0660"
\end{verbatim}

(If you don't need access for the group remove the \texttt{GROUP}
 equation and adjust the mode.)
This line is put into a file, where udev will find it, for example
\verb|/etc/udev/rules.d/80-uinput.rules|.

\subsection{Configuration}
The default configuration file is looked for in 
\verb|$HOME/.config/hid2out.conf|. But you can give any other 
desired path as first argument on the command line.

We use libconfig to parse the file so for basic syntax please 
refer to the documentation of the library.

The following settings have to be contained in the  file:
\begin{description}
\item[input\_devices]
  A list of strings is expected. Each item in the list 
  has to be a path to an input device node.
  With this list you select the hardware that is watched by
  hid2out.
\item[eventmaps]
  A list of settings is expected.
  Each setting associates a name to an event map, which
  defines the actions that are triggered for particular input events.
  At least a map named \texttt{main} must be present.
  This is the map that is in effect initially.
  (Later you can switch to another map.)

  An input map is a list of rules.
  A rule is a list of a pattern and an action.
  For each input event the rules are tried in the order they occur in the map.
  The pattern is matched against the channel event and if it succeeds, the 
  action is executed.
  A channel event consists of four attributes:
  \begin{itemize}
  \item channel
  \item type
  \item code
  \item value
  \end{itemize}
  The latter three correspond to the slots with the same name of 
  the struct \texttt{unput\_event} in \texttt{linux/input.h}.
  The first one identifies the input device from which the event has been
  read. They are numbered according to the paths in the input\_devices setting.
  The first element is associated with channel number 0, the second with channel
  number 1, and so on.
  The channel event matches with the pattern if all of its attributes
  are within the ranges.
  
  Ranges are represented by lists of length 0 to 2
  containing integers and empty lists.
  The first element indicates the minimum,
  the second one the maximum, provided that they are not missing.
  A bound of the range is also considered missing if it is not integers but 
  an empty list.

  You may use evtest to help you figure out how the events 
  triggered by your input hardware 
  are represented internally.

  An action is a list with an action name, a string, 
  as its first element followed by arguments or
  it is an empty list.
 
  These are the actions that are understood:

  \begin{description}
  \item[exit] (no arguments):
    Terminate hid2out.
  \item[switch] with an event map name as argument:
    Switch the current event map to the named one.
  \item[send] followed by 4 integers/empty lists:
    Write a representation of the channel event to the 
    standard output. The four arguments replace the
    attributes of the original channel event,
    in case they are integers and not missing.
    Empty lists are considered missing arguments.
  \item[exec] followed by a string:
    Lauch the process indicated by the string, i.e. the linux standard library function system is 
    called with string as its argument.
  \item[An empty list]:
    Do nothing, just ignore the event.
  \end{description}
  
  The default action is send. So if you only want an event
  to be passed to the standard output you don't have to
  give a rule for that.

  You should at least define a rule with the action ``exit''.
  Otherwise, you may not be able to terminate the program.
  Note, that Ctrl-C may not work, if it is grabbed by 
  hid2out and your shell will not see it.
\end{description}

Let's have a look at an example configuration file

\begin{verbatim}
input_devices =
	      (	"/dev/input/by-id/manufacturer-a-event-kbd",
	        "/dev/input/by-id/manufacturer-b-event-kbd")

eventmaps = { main = (
	      	 #if division key on num pad of first keyboard released,
		 # stop and go back to normal
	      	 ((1, 1, 98, 0), ("exit")),
	      	 #but when pressed, ignore
	      	 ((1, 1, 98, 1), ()),
 	      	 #if + on num pad released, sitch on screen of raspi
	      	 (((), 1, 78, 0), ("exec", "xrandr --output VGA1 --off")),
	      	 #when pressed, ignore
	      	 (((), 1, 78, 1), ()),
 	      	 #if - on num pad released, switch off screen of raspi
	      	 (((), 1, 74, 0), ("exec", "xrandr --output VGA1 --auto")),
	      	 #press - on num pad 
	      	 (((), 1, 74, 1), ()),
	      	 #when calculator key is released, switch to the other eventmap below
	      	 (((), 1, 140, 0), ("switch", "calc")),
	      	 #ignore again
	      	 (((), 1, 140, 1), ())
	      	 ),

	      	 #With the calculator key the special functions of the other keys
	      	 #can be toggled on an off.
	      calc = (
	      	 #release calculator key
	      	 (((), 1, 140, 0), ("switch", "main")),
	      	 #press calculator key
	      	 (((), 1, 140, 1), ())
	      	 )   
	      	 }  
\end{verbatim}

Here two input devices are grabbed by hid2out. 
You can stop the program by pressing the division sign key 
on the numeric key pad.
With plus and minus key, the VGA output is activated resp.~deactivated
with the help of xrandr.
The calculator key can be used to deactivate the special functionality
of the other keys. So their events will be forwarded to standard output.
Only the calculator key is now special and is used to switch back
to the initial event map.

\subsection{Example Use Scenarios}
As already illustrated in the introduction, one example use case is
replicating the input devices on remote machines. For example just use
the keyboard attached to one computer as input device for another machine.
You
can use the big keyboard attached to your desktop with your laptop
without having to unplug and plug every time you change from desktop to
laptop.

The events are represented in a way that is human readable, or let's
better put it nerd readable. So you could be able to put your own tools
in between to modify the event stream or have useful things being
triggered by certain events.

We  avoided the pitfalls of implementing a secure protocol.
So we better rely on ssh for that and you could connect your local
input devices to a remote system entering this line:
\begin{verbatim}
hid2out | ssh <username>@<host> in2hid
\end{verbatim}

Or if you defined the user \texttt{uinput} as above with 
in2hid as shell, you can use this command line:
\begin{verbatim}
hid2out | ssh uinput@<host>
\end{verbatim}


You may also want to use huipfui locally if you need to intercept keyboard or
mouse input. 
\begin{itemize}
\item Define keyboard shortcuts. 
  Desktop environments already allow you to define your keyboard shortcuts.
  But with huipfui you get some advantages:
  \begin{itemize}
  \item Shortcuts also work in text console mode
  \item You only need to define them once and not for each desktop environment
    you are using.
  \item You can define shortcuts for keys your desktop environment is not 
    able to handle, like for example number pad keys in KDE
  \item You can even define shortcuts for mouse buttons.
  \item You can even define shortcuts for switches you didn't know they existed.
    Did you know that some headphone and microphone sockets are linked
    to linux input device nodes? So you can have useful things triggered
    when plugging in a headphone or a microphone.
  \item You need not reconfigure your shortcuts, if your desktop environment 
    increments its version number.
  \end{itemize}
\item Escape applications that grab the human user interface devices
  Some applications running directly on frame buffer grab all input devices.
  Then the key combination for switching to another virtual terminal is 
  blocked and you cannot terminate the application if it is hung.
  With huipfui you can grab the input before starting the application and
  define a key to kill the application.
\item Temporary work around a broken keyboard.
\end{itemize}
For local use you could use the following command line:
\begin{verbatim}
 hid2out | in2hid
\end{verbatim}


\subsection{License}\label{license}

The huipfui project is open-sourced software licensed under the
\href{https://opensource.org/licenses/GPL-3.0}{GNU General Public
License v 3.0}

%Local Variables:
%TeX-master: "readme_frame.tex"
%End:
